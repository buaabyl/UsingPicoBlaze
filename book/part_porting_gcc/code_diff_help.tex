\begin{textcode}
Usage: diff [OPTION]... FILE1 FILE2

  -i  --ignore-case  Consider upper- and lower-case to be the same.
  -w  --ignore-all-space  Ignore all white space.
  -b  --ignore-space-change  Ignore changes in the amount of white space.
  -B  --ignore-blank-lines  Ignore changes whose lines are all blank.
  -I RE  --ignore-matching-lines=RE  Ignore changes whose lines all match RE.
  --binary  Read and write data in binary mode.
  -a  --text  Treat all files as text.

  -c  -C NUM  --context[=NUM]  Output NUM (default 2) lines of copied context.
  -u  -U NUM  --unified[=NUM]  Output NUM (default 2) lines of unified context.
    -NUM  Use NUM context lines.
    -L LABEL  --label LABEL  Use LABEL instead of file name.
    -p  --show-c-function  Show which C function each change is in.
    -F RE  --show-function-line=RE  Show the most recent line matching RE.
  -q  --brief  Output only whether files differ.
  -e  --ed  Output an ed script.
  -n  --rcs  Output an RCS format diff.
  -y  --side-by-side  Output in two columns.
    -w NUM  --width=NUM  Output at most NUM (default 130) characters per line.
    --left-column  Output only the left column of common lines.
    --suppress-common-lines  Do not output common lines.
  -DNAME  --ifdef=NAME  Output merged file to show `#ifdef NAME' diffs.
  --GTYPE-group-format=GFMT  Similar, but format GTYPE input groups with GFMT.
  --line-format=LFMT  Similar, but format all input lines with LFMT.
  --LTYPE-line-format=LFMT  Similar, but format LTYPE input lines with LFMT.
    LTYPE is `old', `new', or `unchanged'.  GTYPE is LTYPE or `changed'.
    GFMT may contain:
      %<  lines from FILE1
      %>  lines from FILE2
      %=  lines common to FILE1 and FILE2
      %[-][WIDTH][.[PREC]]{doxX}LETTER  printf-style spec for LETTER
        LETTERs are as follows for new group, lower case for old group:
          F  first line number
          L  last line number
          N  number of lines = L-F+1
          E  F-1
          M  L+1
    LFMT may contain:
      %L  contents of line
      %l  contents of line, excluding any trailing newline
      %[-][WIDTH][.[PREC]]{doxX}n  printf-style spec for input line number
    Either GFMT or LFMT may contain:
      %%  %
      %c'C'  the single character C
      %c'\OOO'  the character with octal code OOO

  -l  --paginate  Pass the output through `pr' to paginate it.
  -t  --expand-tabs  Expand tabs to spaces in output.
  -T  --initial-tab  Make tabs line up by prepending a tab.

  -r  --recursive  Recursively compare any subdirectories found.
  -N  --new-file  Treat absent files as empty.
  -P  --unidirectional-new-file  Treat absent first files as empty.
  -s  --report-identical-files  Report when two files are the same.
  -x PAT  --exclude=PAT  Exclude files that match PAT.
  -X FILE  --exclude-from=FILE  Exclude files that match any pattern in FILE.
  -S FILE  --starting-file=FILE  Start with FILE when comparing directories.

  --horizon-lines=NUM  Keep NUM lines of the common prefix and suffix.
  -d  --minimal  Try hard to find a smaller set of changes.
  -H  --speed-large-files  Assume large files and many scattered small changes.

  -v  --version  Output version info.
  --help  Output this help.

If FILE1 or FILE2 is `-', read standard input.
\end{textcode}

