\chapter{Introduction}
转眼自己工作四年了,在这四年里做过的东西太杂,涉及到的技术太多,
如果不整理一下过几年就会渐渐的忘记,那么这四年就只是经历而不能成为经验。
我在学校里学的是芯片设计,学的主要是电子系的课程,
没有系统的学过计算机系的课程。毕业后主要做的是嵌入式系统开发,
先分析uIP和lwip的实现,再以此为基础实现了一个专用的协议栈
(属于令牌网,要求实时通信)。而后转作FPGA的设计和开发、
顺带编写上位机驱动,编写过各种脚本。

曾经有个Palm TT5,上面有个圣经软件BiblePlus,
我发现电子版的圣经数据有些错误,于是自己抽空分析Bibleplus的数据格式,
并用c写了一个解析程序,可以解码圣经,修正,然后再编码。可惜后来我的Palm坏了。
于是把代码共享到Google Code,居然有人发邮件联系我提交Bug!
说实在真没想到自己的代码居然有人觉得有用。
但悲剧的是自己看不懂自己的代码了:(

后来自己把CF卡DIY为固态硬盘(那个时候还没有廉价的SSD),
于是做了EWF、FBWF的界面,也有人觉得有用,发邮件给我提需求。
经过这些备受鼓励,于是就想着能不能写点东西。

现在应该就是一个契机,自己想弄一个以太网通信控制器,
实现轻量级的TCPIP协议栈。因为目的是梳理知识,而不是做产品,
所以没必要考虑用ARM来做。

我的想法是先阅读PicoBlaze软核,再从底层往上走。
梳理出最简单的CPU设计方法,可以总结出一些基本的原则,
可以方便的编写符合自己需求的CPU,然后是添加调试接口,
工具链移植,通信接口。最后用多个CPU核实现一个“硬件”TCPIP协议栈。
\\

我从PicoBlaze的使用开始写起,分析它的指令集,如何使用。
如何写一个简单的汇编处理器,自己比较熟悉Python,
所以就用简单的Python脚本来做。

然后开始分析用原语写的PicoBlaze,并把它改写为RTL。
顺带写个汇编预处理器,可以支持一些简单的宏。
再给PicoBlaze添加Jtag调试接口,这个之前没做过,
所以会比较慢,还不知道会弄多久。
移植GCC到新的平台是一件很庞大的工程,
如果要自己写微处理器这一步没法省略的,
所以先从分析别人的工作开始,学习移植步骤,
然后再尝试自己移植。
SPI和UART是常见的两种外设接口,所以在这里一并实现。
后续会计划在上面的基础上写协议栈,可能还要写一个简单的SDRAM控制器。
毕竟网络的数据量是比较大的,光用Block RAM可能不够。
\\

本书打算在今年写完,用\LaTeXe来排版,中英文交替着写,也练习英文写作。
这就当复习如何写论文了:)

\clearpage
\section{KCPSM3}
$PicoBlaze^{\textregistered{}}$是$Xilinx^{\textregistered{}}$针对低端应用开发的8位处理器,
开放源代码。

PicoBlaze另一个名称KCPSM,是Constant(K) Coded Programmable State Machine的简称,
也即常量编码状态机。由Ken Chapman开发,所以也有人这么理解Ken Chapman's PSM。

本文分析的主要是Spartan3系列的版本,即KCPSM3,
典型的KCPSM3占用96个Slices,等价于低端XC3S200器件5\%的资源。

PicoBlaze是8位处理器,使用18位指令集,每个指令执行时间为2个周期。 
KCPSM3直接用xilinx原语编写,只有汇编器。这样就导致在使用中有些问题,
其一汇编维护起来太麻烦,其次想添加新的指令很难。

而PicoBlaze只有一份很旧的设计说明,还是针对停产的VertixII器件的。
这篇文章是“Creating Embedded Microcontrollers (Programmable State Machines)”,
很值得一读,不过还是旧了一点。
PicoBlaze缺少一个完整的设计文档,当然也没找到KCPSM3的设计说明。
所以想要修改它其实非常不容易,我准备阅读它的源码,归纳设计的方法。
虽然已经有现成的PacoBlaze\footnote{PacoBlaze:一个基于PicoBlaze的开源实现},
但想要方便以后的使用,还是自己阅读源码,并改写为Verilog。
再次说一句我的目的是梳理知识:)

\clearpage
\section{Application of PicoBlaze}
\begin{enumerate}
\item LED flasher.
\item PWM control and even generation.
\item Switch monitor.
\item UART interface and simple command/status terminal.
\item LCD character module display interface and control.
\item SPI master
\item I2C master.
\item Calculator.
\item Audio DSP processor.
\item DTMF tone telephone dialer including sine wave generation.
\item System monitoring.
\item Motor control.
\item Rotary encoder interface.
\item Calculator for frequency synthesizer.
\item Calculator for filter coefficient generation.
\item Emulation of a different micro controller.
\item PID control.
\item Mouse/Keyboard interface.
\item Keypad scanner.
\item Power supply monitoring and control.
\item Servo control.
\item Built-in test equipment.
\item Configuration management.
\item Design Authentication Processor.
\item Implementing peripherals for MicroBlaze or PPC.
\item Interrupt controller for MicroBlaze or PPC.
\end{enumerate}

\section{Code style}
\textbf{c code style:}
\begin{ccode}
int adder(int a, int b)
{
    return a + b;
}
\end{ccode}

\textbf{verilog code style:}
\begin{vcode}
module adder(input a, 
        input b,
        output c);

assign c = a + b;

endmodule
\end{vcode}

