\chapter{Introduction}
%brief
$PicoBlaze^{\textregistered{}}$是$Xilinx^{\textregistered{}}$针对低端应用开发的8位处理器,
开放源代码。

\section{KCPSM3}
PicoBlaze另一个名称KCPSM,是Constant(K) Coded Programmable State Machine的简称,
也即常量编码状态机。由Ken Chapman开发,所以也有人这么理解Ken Chapman's PSM。

本文分析的主要是Spartan3系列的版本,即KCPSM3,
典型的KCPSM3占用96个Slices,等价于低端XC3S200器件5\%的资源。

PicoBlaze是8位处理器,使用18位指令集,每个指令执行时间为2个周期。 
KCPSM3直接用xilinx原语编写,只有汇编器。这样就导致在使用中有些问题,
其一汇编维护起来太麻烦,其次想添加新的指令很难。

而PicoBlaze的只有一份很旧的设计说明,针对VertixII器件的
“Creating Embedded Microcontrollers (Programmable State Machines)”,
缺少一个完整的设计文档,当然也没找到KCPSM3的设计说明。
想要修改它非常不容易,所以我准备阅读它的源码,归纳设计的方法。
虽然已经有PacoBlaze\footnote{PacoBlaze:一个基于PicoBlaze的开源实现},
但想要方便以后的使用,还是自己阅读源码,并改写为Verilog。
也是为了开阔自己的思路,提高能力。

\clearpage
\section{Application of PicoBlaze}
\begin{enumerate}
\item LED flasher.
\item PWM control and even generation.
\item Switch monitor.
\item UART interface and simple command/status terminal.
\item LCD character module display interface and control.
\item SPI master
\item I2C master.
\item Calculator.
\item Audio DSP processor.
\item DTMF tone telephone dialer including sine wave generation.
\item System monitoring.
\item Motor control.
\item Rotary encoder interface.
\item Calculator for frequency synthesizer.
\item Calculator for filter coefficient generation.
\item Emulation of a different micro controller.
\item PID control.
\item Mouse/Keyboard interface.
\item Keypad scanner.
\item Power supply monitoring and control.
\item Servo control.
\item Built-in test equipment.
\item Configuration management.
\item Design Authentication Processor.
\item Implementing peripherals for MicroBlaze or PPC.
\item Interrupt controller for MicroBlaze or PPC.
\end{enumerate}

\section{Code style}
\textbf{c code style:}
\begin{ccode}
int adder(int a, int b)
{
    return a + b;
}
\end{ccode}

\textbf{verilog code style:}
\begin{vcode}
module adder(input a, 
        input b,
        output c);

assign c = a + b;

endmodule
\end{vcode}

